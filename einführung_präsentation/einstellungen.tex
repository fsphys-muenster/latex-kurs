\usepackage[no-math]{fontspec}
\usepackage{polyglossia}
\setdefaultlanguage{german}

\usepackage{silence}
\WarningFilter{csquotes}{Using preliminary 'polyglossia' interface.}
% biblatex-Warnung harmlos, footnote-patching wird nur für „ibid“-Stile gebraucht
% vgl. https://tex.stackexchange.com/questions/202988/beamer-patching-footnotes-warning
%      https://tex.stackexchange.com/questions/108511/gb4e-biblatex-error-patching-footnotes-failed
\WarningFilter{biblatex}{Patching footnotes failed.}

\usepackage{microtype}
\usepackage{selnolig}

\usepackage{xcolor}
\usepackage{gincltex}
\usepackage{wrapfig}
\usepackage{grffile}
% subcaption (noch?) inkompatibel mit beamer!
% => „compatibity=false“ für subcaption-Paket notwending
% Hier als Ersatz subfig mit „caption=false“ verwendet
% vgl. https://tex.stackexchange.com/questions/125579/subcaption-with-beamer
%      https://tex.stackexchange.com/questions/31906/subcaption-package-compatibility-issue
\usepackage[caption=false]{subfig}

\usepackage{hologo}
\usepackage{csquotes}
\usepackage[useregional]{datetime2}
\usepackage{siunitx}
\usepackage{ragged2e}
\usepackage{adjustbox}

\usepackage{xfrac}
\usepackage{tikz}
\usepackage{listingsutf8}
\usepackage{lstautogobble}
\usepackage[backend=biber, style=authortitle]{biblatex}

\usepackage{hyperref}
\usepackage{cleveref}


% --- Beamer-Einstellungen
\usefonttheme{serif} % only math
\setbeamertemplate{caption}[numbered]
\usetheme[secheader]{Boadilla}
\beamertemplatenavigationsymbolsempty

% --- Paket-Einstellungen
% -- csquotes
\MakeOuterQuote{"}
% -- hyperref
% nur URLs einfärben
\colorlet{linkcolor}{blue!35!black}
\hypersetup{unicode, colorlinks, linkcolor=, urlcolor=linkcolor}
% -- siunitx
\sisetup{
	locale=DE,
	separate-uncertainty,
	quotient-mode=fraction,
	per-mode=fraction,
	fraction-function=\sfrac,
	detect-all
}
% Notwendig, damit das \micro (µ) die richtige Schriftart hat
\AtBeginDocument{\sisetup{math-rm=\mathrm, text-rm=\rmfamily}}
% -- tikz
\usetikzlibrary{shapes.geometric,tikzmark,fit}
% -- listings
\let\listingsfont\ttfamily
\colorlet{keywordcolor}{purple!60!black}
\colorlet{emphcolor}{teal!70!black}
\colorlet{commentcolor}{green!40!black}
\colorlet{texcsicolor}{blue}
\colorlet{texcsiicolor}{blue!50!black}
\colorlet{bracescolor}{orange!70!black}
\colorlet{dollarcolor}{red}
\definecolor{stringcolor}{RGB}{47, 0, 255}
\lstdefinestyle{mystandard}{
	% Stelle UTF8-Input mit Latin1 dar
	%inputencoding=utf8/latin1,
	% Zeilennummern links
	%numbers=left,
	% Darstellung der Zeilennummern
	numberstyle=\listingsfont,
	% Rahmen (links)
	%frame=L,
	columns=flexible,
	%xleftmargin=0.5cm,
	% Default Schriftart/Größe
	basicstyle=\small\listingsfont,
	% Darstellung für Keywords
	keywordstyle=\color{keywordcolor},
	% Darstellung für betonte Bezeichner
	emphstyle=\color{emphcolor},
	% Darstellung für Kommentare
	commentstyle=\color{commentcolor},
	% Darstellung für Strings
	stringstyle=\color{stringcolor},
	% Darstellung für TeX-Befehle; Stern: mit backslash
	texcsstyle=*\color{texcsicolor},
	texcsstyle=*[2]\color{texcsiicolor},
	% Tab-Breite
	tabsize=4,
	% Leerzeichen in Strings hervorheben
	showstringspaces=false,
	% Tabs hervorheben
	showtabs=false,
	% Leerzeichen hervorheben
	showspaces=false,
	% Alle Leerzeichen behalten
	keepspaces=true,
	% „Normale“ Anführungszeichen: ""
	upquote=true,
	% Lange Zeilen aufbrechen
	breaklines=true,
	% Zeilen an Whitespace aufbrechen
	breakatwhitespace=true,
	% Führende Einrückung automatisch entfernen
	autogobble,
}

\lstset{
	language=[LaTeX]{TeX},
	style=mystandard,
	escapechar=|,
	deletetexcs={begin, end, documentclass, usepackage},
	moretexcs=[2]{begin, end, documentclass, usepackage},
	moretexcs={
		f, latex, Alpha, eps,
		part, chapter, subsection, subsubsection, paragraph, subparagraph, color, text, mathbb, mathcal, includegraphics, cref, autocite, footcite,
		SI, num, si, SIrange, SIlist, m, per, s, squared, cm, nm, MHz, mega, Hz, cubed, tothe, mA, kV
	},
	literate=
		{\$}{\textcolor{dollarcolor}{\$}}1
		{\{}{\textcolor{bracescolor}{\{}}1
		{\}}{\textcolor{bracescolor}{\}}}1,
}

% --- Neue/eigene Befehle
\let\strong\textbf
\newcommand{\email}[1]{\href{mailto:#1}{\texttt{#1}}}
\newcommand*\circled[1]{\tikz[baseline=(char.base)]{\node[shape=ellipse, draw, inner sep=2pt] (char) {#1};}}

