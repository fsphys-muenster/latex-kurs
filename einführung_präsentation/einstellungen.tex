% --- Paket-Einstellungen
% -- csquotes
\MakeOuterQuote{"}
% -- hyperref
\hypersetup{unicode}
% -- siunitx
\sisetup{
	locale=DE,
	separate-uncertainty,
	quotient-mode=fraction,
	per-mode=fraction,
	fraction-function=\sfrac,
	detect-all
}
\AtBeginDocument{\sisetup{math-rm=\mathrm, text-rm=\rmfamily}}
% -- listings
\colorlet{keywordcolor}{purple!60!black}
\colorlet{commentcolor}{green!40!black}
\colorlet{texcsicolor}{blue}
\colorlet{texcsiicolor}{blue!50!black}
\colorlet{bracescolor}{orange!70!black}
\colorlet{dollarcolor}{red}
\lstdefinestyle{mystandard}{
	% Stelle UTF8-Input mit Latin1 dar
	%inputencoding=utf8/latin1,
	% Zeilennummern links
	%numbers=left,
	% Darstellung der Zeilennummern
	numberstyle=\ttfamily,
	% Rahmen (links)
	%frame=L,
	columns=flexible,
	%xleftmargin=0.5cm,
	% Default Schriftart/Größe
	basicstyle=\small\ttfamily,
	% Darstellung für Keywords
	keywordstyle=\color{keywordcolor},
	% Darstellung für Kommentare
	commentstyle=\color{commentcolor},
	% Darstellung für Strings
	stringstyle=\color[RGB]{ 47,   0, 255},
	% Darstellung für TeX-Befehle; Stern: mit backslash
	texcsstyle=*\color{texcsicolor},
	texcsstyle=*[2]\color{texcsiicolor},
	% Tab-Breite
	tabsize=4,
	% Leerzeichen in Strings hervorheben
	showstringspaces=false,
	% Tabs hervorheben
	showtabs=false,
	% Leerzeichen hervorheben
	showspaces=false,
	% Alle Leerzeichen behalten
	keepspaces=true,
	% "Normale" Anführungszeichen: ""
	upquote=true,
	% Lange Zeilen aufbrechen
	breaklines=true,
	% Zeilen an Whitespace aufbrechen
	breakatwhitespace=true,
}

\lstset{
	language=[LaTeX]{TeX},
	style=mystandard,
	escapechar=|,
	deletetexcs={begin, end, documentclass, usepackage},
	moretexcs=[2]{begin, end, documentclass, usepackage},
	moretexcs={
		f, latex, Alpha,
		part, chapter, subsection, subsubsection, paragraph, subparagraph, color, text, mathbb, mathcal, includegraphics, cref, autocite, footcite,
		SI, num, si, SIrange, SIlist, m, per, s, squared, cm, nm, MHz, mega, Hz, cubed, tothe, mA, kV
	},
	literate=
		{\$}{\textcolor{dollarcolor}{\$}}1
		{\{}{\textcolor{bracescolor}{\{}}1
		{\}}{\textcolor{bracescolor}{\}}}1,
}
% --- Befehle
\newcommand{\email}[1]{\href{mailto:#1}{\texttt{#1}}}
