% Autor: Simon May
% Datum: 2014-08-16

% Befehl, um die E-Mail-Adressen darzustellen
\makeatletter
\newcommand{\protokollemailparse}[1]{
	\@for\@tempa:=#1\do{
		\email{\@tempa}\par
	}
}
\makeatother

% Informationen für die PDF-Datei
\hypersetup{
	pdfinfo={
		Title={Versuchsprotokoll \varNum. \varName},
		Author={\varAutor}
	}
}

\begin{titlepage}
	\vspace*{2.5cm}
	\begin{center}
		\Huge
		\textbf{Versuchsprotokoll \varNum}

		\LARGE
		\varName

		\vspace{0.5cm}
		\large
		\varDate

		\IfFileExists{res/titelseite.pdf}{
			\vspace{0.5cm}
			\includegraphics[0.5\textwidth]{res/titelseite.pdf}
		}{
		\vspace{4cm}
		}

		\vspace{1.5cm}
		\varAutor

		\vspace{1cm}
		\normalsize
		\varGruppe

		\ifthenelse{\boolean{showEmail}}{
			\protokollemailparse{\varEmail}
		}{}  
	\end{center}
\end{titlepage}

