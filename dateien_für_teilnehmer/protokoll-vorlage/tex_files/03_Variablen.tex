% Autor: Simon May
% Datum: 2015-08-05

% Der Befehl \newcommand kann auch benutzt werden, um Variablen zu definieren:

% Nummer des Versuchs (z.B. M2):
\newcommand{\varNum}{M1}
% Name des Versuchs:
\newcommand{\varName}{Spektrometer}
% Name des Versuchs (kurz, z.B. für Kopfzeile):
\newcommand{\varNameShort}{Spektrometer}
% Ist der Versuchstitel sehr lang? (Verringert Schriftgröße des Titels, falls
% "true")
\newcommand{\varLongTitle}{false}
% Datum der Durchführung:
\newcommand{\varDate}{\formatdate{24}{6}{2015}}
% Autoren des Protokolls:
\newcommand{\varAuthor}{Simon May, Max Mustermann}
% Nummer der eigenen Gruppe:
\newcommand{\varGroup}{Gruppe 14}
% E-Mail-Adressen der Autoren (kommagetrennt ohne Leerzeichen!):
\newcommand{\varEmail}{simon.may@uni-muenster.de,m\_must08@uni-muenster.de}
% E-Mail-Adressen anzeigen (true/false):
\newcommand{\varShowEmail}{true}
% Kopfzeile anzeigen (true/false):
\newcommand{\varShowHeader}{true}
% Inhaltsverzeichnis anzeigen (true/false):
\newcommand{\varShowTOC}{true}
% Literaturverzeichnis anzeigen (true/false):
\newcommand{\varShowBibliography}{true}

