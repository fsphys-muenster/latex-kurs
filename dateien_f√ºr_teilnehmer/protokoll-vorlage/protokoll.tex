% LaTeX-Vorlage für Versuchsprotokolle
% Autor: Simon May
% Datum: 2016-10-13

% Es gibt die Dokumenttypen scrartcl („Artikel“), scrreprt („Bericht“),
% scrbook („Buch“) und scrlttr2 („Brief“). Diese gehören zum KOMA-Script,
% bieten mehr Optionen als die „Standardklassen“ und sollten besonders für
% deutsche Texte benutzt werden.
% Natürlich gibt es noch weitere Klassen, z.B. beamer für Präsentationen.
\documentclass[
	% Sprache für z.B. Babel
	ngerman,
	% Papierformat
	a4paper,
	% Einseitig
	oneside,
	% Zweiseitig
	%twoside,
	% Zweispaltig
	%twocolumn,
	% Schriftgröße (beliebige Größen mit „fontsize=Xpt“)
	12pt,
	% schreibt die Papiergröße korrekt ins Ausgabedokument
	pagesize,
	% größere Kopfzeile (wegen LaTeX-Warnung)
	headlines=1.4,
	% Literaturverzeichnis unnummeriert im Inhaltsverzeichnis aufführen
	toc=bibliography,
	% Abbildungsverzeichnis etc. unnummeriert im Inhaltsverzeichnis aufführen
	toc=listof,
	% \caption bei Tabellen immer als Überschrift setzen
	captions=tableheading,
	automark
]{scrartcl}

% --- Pakete einbinden
% --- Pakete einbinden
% --- Pakete erweitern LaTeX um zusätzliche Funktionen. Dies ist ein Satz nützlicher Pakete.

% Silbentrennung; Sprache wird durch Option bei \documentclass festgelegt
\usepackage{babel}
% Verwendung der Zeichentabelle T1 (Sonderzeichen etc.)
\usepackage[T1]{fontenc}
% Legt die Zeichenkodierung der Eingabedatei fest, z.B. UTF8
\usepackage[utf8]{inputenc}
% Schriftart
\usepackage{lmodern}

% Einige LaTeX-Bugfixes
\usepackage{fixltx2e}
% Nutzen von +, -, *, / in \setlength u.ä. (z.B. \setlength{\a + 3cm})
\usepackage{calc}
% Wird benötigt, um \ifthenelse zu benutzen
\usepackage{xifthen}
% Optionen für eigene definierte Befehle
\usepackage{xparse}

% Verbessertes Aussehen von Text
\usepackage{microtype}
% Automatische Anführungszeichen
\usepackage[autostyle]{csquotes}
% Wird für Kopf- und Fußzeile benötigt
\usepackage{scrpage2}
% Einfaches Wechseln zwischen unterschiedlichen Zeilenabständen
\usepackage{setspace}
% Optionen für Listen (enumerate, itemize, ...)
\usepackage{enumitem}
% Zusätzliche Optionen für Tabellen (tabular)
\usepackage{array}

% Mathepaket (intlimits: Grenzen über/unter Integralzeichen)
\usepackage[intlimits]{amsmath}
% Mathe-Symbole, \mathbb etc.
\usepackage{amssymb}
% Weitere Mathebefehle
\usepackage{mathtools}
% "Schöne" Brüche im Fließtext
\usepackage{xfrac}
% Ermöglicht die Nutzung von \SI{Zahl}{Einheit} u.a.
\usepackage{siunitx}

% Farben
\usepackage{xcolor}
% Zum flexiblen Einbinden von Grafiken (\includegraphics)
\usepackage{graphicx}
% .tex-Dateien mit \includegraphics einbiden
\usepackage{gincltex}
% Abbildungen im Fließtext
\usepackage{wrapfig}
% Zitieren, Bibliographie
\usepackage[style=verbose, backend=biber]{biblatex}
% Darstellung von Captions
\usepackage[font=small, labelfont=bf, format=plain]{caption}
% Abbildungen nebeneinander
\usepackage{subcaption}

% Verlinkt Textstellen im PDF-Dokument
\usepackage[pdfpagelabels, unicode]{hyperref}
% "Schlaue" Referenzen (nach hyperref laden!)
\usepackage{cleveref}



% --- Einstellungen
% -- latex
% größere Kopfzeile (wegen LaTeX-Warnung)
\setlength{\headheight}{1.4\baselineskip}
% 1,5-facher Zeilenabstand
\onehalfspacing

% -- biblatex (Literaturverzeichnis)
\IfFileExists{res/literatur.bib}{
	\addbibresource{res/literatur.bib}
}{}

% -- csquotes
% Anführungszeichen automatisch umwandeln
\MakeOuterQuote{"}

% -- siunitx
\sisetup{
	locale=DE,
	separate-uncertainty,
	input-product=*,
	output-product=\cdot,
	quotient-mode=fraction,
	per-mode=fraction,
	fraction-function=\sfrac
}

% -- hyperref
\hypersetup{
	% Links/Verweise mit Kasten der Dicke 0.5pt versehen
	pdfborder={0 0 0.5}
}

% -- cleveref
\crefformat{footnote}{#2\textsuperscript{#1}#3}



% --- Eigene Befehle einbinden
% Autor: Simon May
% Datum: 2015-08-05

% Eigene Befehle eignen sich gut, um Abkürzungen für lange Befehle zu erstellen.
% Die Syntax ist folgende:
% \newcommand{neuer Befahl}[Anzahl Parameter (optional)]{ein langer Befehl}
% Das folgende Beispiel fügt ein Bild mit bestimmten vorgegebenen Optionen ein:
\newcommand{\centeredImage}[1]{
	\begin{figure}[h!]
		\centering
		\includegraphics[width=0.50\textwidth]{#1}
	\end{figure}
}
% #1 ist dabei ein Parameter, den man \centeredImage übergeben muss. In
% 10_Titelseite.tex wird dieser Befehl verwendet. Der Parameter ist dort
% Bilder/titelseite.jpg.
% Benötigt man keine Parameter, dann lässt man [1] weg. Werden zusätzliche
% Parameter benötigt, dann kann man die Zahl auf maximal 9 erhöhen.

% \arsinh etc.
\newcommand{\arsinh}{\operatorname{arsinh}}
\newcommand{\arcosh}{\operatorname{arcosh}}
\newcommand{\artanh}{\operatorname{artanh}}

% Ein Befehl, um eine E-Mail-Adresse darzustellen bzw. automatisch zu verlinken
\newcommand{\email}[1]{\href{mailto:#1}{\texttt{#1}}}

% Befehl, um die E-Mail-Adressen auf der Titelseite darzustellen
\makeatletter
\newcommand{\protokollemailparse}[1]{
	\@for\@tempa:=#1\do{
		\email{\@tempa}\par
	}
}
\makeatother


% --- Variablen importieren
% Autor: Simon May
% Datum: 2015-08-05

% Der Befehl \newcommand kann auch benutzt werden, um Variablen zu definieren:

% Nummer des Versuchs (z.B. M2):
\newcommand{\varNum}{M1}
% Name des Versuchs:
\newcommand{\varName}{Spektrometer}
% Name des Versuchs (kurz, z.B. für Kopfzeile):
\newcommand{\varNameShort}{Spektrometer}
% Ist der Versuchstitel sehr lang? (Verringert Schriftgröße des Titels, falls
% "true")
\newcommand{\varLongTitle}{false}
% Datum der Durchführung:
\newcommand{\varDate}{\formatdate{24}{6}{2015}}
% Autoren des Protokolls:
\newcommand{\varAuthor}{Simon May, Max Mustermann}
% Nummer der eigenen Gruppe:
\newcommand{\varGroup}{Gruppe 14}
% E-Mail-Adressen der Autoren (kommagetrennt ohne Leerzeichen!):
\newcommand{\varEmail}{simon.may@uni-muenster.de,m\_must08@uni-muenster.de}
% E-Mail-Adressen anzeigen (true/false):
\newcommand{\varShowEmail}{true}
% Kopfzeile anzeigen (true/false):
\newcommand{\varShowHeader}{true}
% Inhaltsverzeichnis anzeigen (true/false):
\newcommand{\varShowTOC}{true}
% Literaturverzeichnis anzeigen (true/false):
\newcommand{\varShowBibliography}{true}


\newboolean{showEmail}
\setboolean{showEmail}{\varShowEmail}
\newboolean{showHeader}
\setboolean{showHeader}{\varShowHeader}
\newboolean{showTOC}
\setboolean{showTOC}{\varShowTOC}
\newboolean{showBibliography}
\setboolean{showBibliography}{\varShowBibliography}

% Kopf- und Fußzeile konfigurieren
\ifthenelse{\boolean{showHeader}}{
	\KOMAoptions{headsepline, DIV=current}
	% Innenseite der Kopfzeile
	\ihead{\headmark}
	% Mitte der Kopfzeile
	\chead{}
	% Außenseite der Kopfzeile
	\ohead{\varAuthor}
}{}
% Innnenseite der Fußzeile
\ifoot{}
% Mitte der Fußzeile
\cfoot{- {\pagemark} -}
% Außenseite der Fußzeile
\ofoot{}

% Metadaten für die PDF-Datei
\hypersetup{
	pdftitle={Versuchsprotokoll: \varName},
	pdfauthor={\varAuthor},
	pdfsubject={Grundpraktikum},
	pdfkeywords={Physik, Münster, Praktikum, Versuchsprotokoll}
}

\begin{document}

% Römische Seitenzahlen für Titelseite/Inhaltsverzeichnis
\pagenumbering{roman}
% Zunächst ohne Kopf-/Fußzeile
\pagestyle{scrplain}

% --- Titelseite einbinden
\IfFileExists{tex_files/04_Titelseite.tex}{
	% Autor: Simon May
% Datum: 2014-08-16

% Befehl, um die E-Mail-Adressen darzustellen
\makeatletter
\newcommand{\protokollemailparse}[1]{
	\@for\@tempa:=#1\do{
		\email{\@tempa}\par
	}
}
\makeatother

% Informationen für die PDF-Datei
\hypersetup{
	pdfinfo={
		Title={Versuchsprotokoll \varNum. \varName},
		Author={\varAutor}
	}
}

\begin{titlepage}
	\vspace*{2.5cm}
	\begin{center}
		\Huge
		\textbf{Versuchsprotokoll \varNum}

		\LARGE
		\varName

		\vspace{0.5cm}
		\large
		\varDate

		\IfFileExists{res/titelseite.pdf}{
			\vspace{0.5cm}
			\includegraphics[0.5\textwidth]{res/titelseite.pdf}
		}{
		\vspace{4cm}
		}

		\vspace{1.5cm}
		\varAutor

		\vspace{1cm}
		\normalsize
		\varGruppe

		\ifthenelse{\boolean{showEmail}}{
			\protokollemailparse{\varEmail}
		}{}  
	\end{center}
\end{titlepage}


}{}

% --- Inhaltsverzeichnis einbinden
\ifthenelse{\boolean{showTOC}}{
	\tableofcontents
	\clearpage
}{}

% Zurücksetzen der Seitenzahlen auf arabische Ziffern
\setcounter{page}{1}
\pagenumbering{arabic}
% Ab hier mit Kopf- und Fußzeile
\pagestyle{scrheadings}

% --- Den Inhalt der Arbeit einbinden
\section{Test-Überschrift}
In diese Datei kommt der eigentliche Inhalt des Protokolls (Theorie, Auswertung, Diskussion etc.).


% --- Anhang einbinden
\IfFileExists{tex_files/20_Anhang.tex}{
	\clearpage
	\appendix
	\section{Anhang}
	\label{sec:anhang}
	\subsection{Fehlerrechnung%
\label{sec:fehlerrechnung}}
Die aufgrund der Unsicherheit in der Zeitmessung entstandenen Messunsicherheiten waren so gering, dass sie bei der Auswertung vernachlässigt wurden. Insbesondere waren sie in der grafischen Darstellung zu klein, um sichtbar zu sein.


Bei der mehreren Messungen derselben Größe (hier: Volumendurchsatz des Kühlwassers) bestimmt sich die Unsicherheit des Mittelwerts aus einem möglichen systematischen Fehler $u$ und dem statistischen Fehler mit der Standardabweichung $s$, der Anzahl der Messungen $n$ und dem Korrekturfaktor $\tau$:
\[\Delta a = u + \frac{\tau s}{\sqrt{n}}\]
Dabei wurde $\tau$ für ein Vertrauensniveau von \SI{95,45}{\%} bzw. $2 \sigma$ gewählt.

%\begin{table}[h]
%	\centering
%	\caption{Messunsicherheiten bei direkt gemessenen Größen.}
%	\label{tab:unsicherheiten}
%	\begin{tabular}{l | l}
%		Größe				& Unsicherheit	\\ \hline\hline
%		$U$ (Akkumulatoren)	& $\Delta U = \SI{0,1}{V}$ \\ \hline
%		$R_a$ (Akkumulatoren)	& $\Delta R_a = \SI{10}{\%}$ \\ \hline
%		$U$ (Gleich-/Wechselstrom)	& $\Delta U = \SI{1}{V}$ \\ \hline
%		$I$ (Gleich-/Wechselstrom)	& $\Delta I = \SI{0,01}{A}$ \\ \hline
%		$P$ (Gleich-/Wechselstrom)	& $\Delta P = \SI{1}{W}$ \\ \hline
%		Wechselspannungsfrequenz $f$	& $\Delta f = \SI{1}{\%}$
%	\end{tabular}
%\end{table}
%
Bei aus Messwerten berechneten Größen muss zur Bestimmung der Unsicherheit der Größe die Fehlerfortpflanzung angewandt werden. Bei den folgenden Größen wurde dies durchgeführt. Es ergibt sich:

\begin{sagesilent}
W_R(c_W, rho, V, f, T) = c_W*rho*V*T/f
var('V', latex_name="V'")
delta_W_R = func_to_expr(gauss_error(W_R))

P_el(U, I) = U * I
delta_P_el = func_to_expr(gauss_error(P_el))

epsilon(W_R, Q_1, Q_2) = abs(Q_2)/abs(Q_1-Q_2+W_R)
delta_epsilon = func_to_expr(gauss_error(epsilon))

eta(W, Q) = abs(W)/abs(Q)
delta_eta = func_to_expr(gauss_error(eta))

P_Z(F, f, r) = 2 * pi * f * F * r
delta_P_Z = func_to_expr(gauss_error(P_Z))
\end{sagesilent}

Für die an das Wasser abgegebene Wärme bzw. die Reibungsarbeit:
\[\Delta Q = \Delta W_R = \sage{delta_W_R}\]


\begin{sagesilent}
P(c_W,m_W,m)=c_W*m_W*m
delta_P = func_to_expr(gauss_error(P))
\end{sagesilent}
Für die Heiz- und Kühlleistung:
\[\Delta P = \sage{delta_P}\]


\begin{sagesilent}
Q_U(P, f)=P/f
delta_Q_U = func_to_expr(gauss_error(Q_U))
\end{sagesilent}
Für die abgegebene Wärme pro Umlauf:
\[\Delta Q_U = \sage{delta_Q_U}\]

Für die Leistungszahl von Kältemaschine und Wärmepumpe:
\begin{align*}
\Delta \epsilon = 
\Bigg(\Delta Q_2^{2} {\left(\frac{Q_{2}}{{\left| Q_{1} - Q_{2} + W_{R} \right|} {\left| Q_{2} \right|}} + \frac{{\left(Q_{1} - Q_{2} + W_{R}\right)} {\left| Q_{2} \right|}}{{\left| Q_{1} - Q_{2} + W_{R} \right|}^{3}}\right)}^{2} \\
+ \frac{\Delta Q_1^{2} {\left(Q_{1} - Q_{2} + W_{R}\right)}^{2} {\left| Q_{2} \right|}^{2}}{{\left| Q_{1} - Q_{2} + W_{R} \right|}^{6}} + \frac{\Delta W_R^{2} {\left(Q_{1} - Q_{2} + W_{R}\right)}^{2} {\left| Q_{2} \right|}^{2}}{{\left| Q_{1} - Q_{2} + W_{R} \right|}^{6}}\Bigg)^{\frac{1}{2}}
\end{align*}

Für die spezifische Wärmekapazität von Eis:
\[\Delta c_E = \sage{func_to_expr(delta_c_E)}\]

Für die elektrische Heizleistung:
\[\Delta P_\text{el} = \sage{delta_P_el}\]

Für den Wirkungsgrad der Wärmekraftmaschine:
\[\Delta \eta = \sage{delta_eta}\]

Für die Leistung beim Pronyschen Zaum:
\[\Delta P_Z = \sage{delta_P_Z}\]



Die jeweiligen berechneten Unsicherheiten sind in der Auswertung zu finden.

}{}

% --- Literaturverzeichnis mit BibLaTeX
\ifthenelse{\boolean{showBibliography}}{
	\clearpage
	\printbibliography
}{}

\end{document}

