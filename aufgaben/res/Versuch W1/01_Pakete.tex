% --- Pakete einbinden
% Pakete erweitern LaTeX um zusätzliche Funktionen. Dies ist ein Satz nützlicher Pakete.
\usepackage{babel}                        % Silbentrennung nach neuer deutscher und englischer Rechtschreibung; Ausgewählt: ngerman (zuletztgenannt)
\usepackage{iftex}
\ifLuaTeX
	\usepackage{fontspec}
	\fontspec{Latin Modern Roman}         % Schriftart
\else
	\usepackage[T1]{fontenc}              % Verwendung der Zeichentabelle T1 (Sonderzeichen etc.)
	\usepackage[utf8]{inputenc}           % Legt die Zeichenkodierung fest, z.B. UTF8
	\usepackage{lmodern}                  % Schriftart
	\usepackage{textcomp}
	\usepackage{fixltx2e}
\fi
\usepackage{microtype}                    % Verbessertes Aussehen von Text
\usepackage{csquotes}                     % Automatische Anführungszeichen
\usepackage[intlimits]{amsmath}           % Mathepaket (Grenzen über/unter Integralzeichen)
\usepackage{amssymb}                      % Symbole, mathbb etc.
\usepackage{mathtools}                    % Weitere Mathebefehle
\usepackage[
	style=verbose-ibid,
	backend=biber
]{biblatex}                               % Zitieren, Bibliographie
\usepackage[pdfpagelabels]{hyperref}      % Verlinkt Textstellen im PDF Dokument
%\usepackage{xcolor}                      % Farben
%\usepackage{tablefootnote}               % Fußnoten in Tabellen
\usepackage[font=small,labelfont=bf,
format=plain]{caption}                    % Darstellung für Caption
\usepackage{subcaption}                   % Bilder nebeneinander
\usepackage{xifthen}                      % Wird benötigt, um \ifthenelse zu benutzen
\usepackage{graphicx}                     % Zum flexiblen Einbinden von Grafiken
\usepackage{siunitx}                      % Ermöglicht die Nutzung von \SI{Zahl}{Einheit}
\usepackage{setspace}                     % Einfaches Wechseln zwischen unterschiedlichen Zeilenabständen
\usepackage{enumitem}                     % Optionen für Listen
\usepackage{wrapfig}                      % Abbildungen im Fließtext
\usepackage{gincltex}                     % .tex-Dateien mit \includegraphics einbiden
\usepackage{scrpage2}                     % Wird für Kopf- und Fußzeile benötigt
\usepackage{sagetex}                      % Sage-Befehle in LaTeX
\usepackage{array}                        % Wird für die Ausrichtung der Tabellenspalten benötigt
\usepackage{dcolumn}                      % Wird für die Ausrichtung der Tabellenspalten benötigt
\usepackage{placeins}
\usepackage{cleveref}                     % "Schlaue" Referenzen
\usepackage{gnuplot-lua-tikz}             % gnuplot-Plots im TikZ-Format einbinden
\usepackage{xparse}                       % Optionen für eigene definierte Befehle
\usepackage{xfrac}                        % Brüche im Fließtext
%\usepackage{breqn}                        % lange Gleichungen aufbrechen

% --- Einstellungen
% latex
\onehalfspacing % 1,5-facher Zeilenabstand
%\setlength{\headheight}{1.1\baselineskip}
%\pdfimageresolution=96

% Literaturverzeichnis (biblatex)
\IfFileExists{res/literatur.bib}{
	\addbibresource{res/literatur.bib}
}

% csquotes
\MakeOuterQuote{"} % Anführungszeichen automatisch umwandeln

% siunitx
\sisetup{
	locale=DE,
	group-digits=false,
	separate-uncertainty,
	output-product=\cdot,
	output-decimal-marker={,},
	input-product=*
}

% cleveref
\crefformat{footnote}{#2\footnotemark[#1]#3}
\crefname{equation}{}{}
\Crefname{equation}{}{}

