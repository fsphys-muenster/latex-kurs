% Eigene Befehle eignen sich gut um Abkürzungen für lange Befehle zu erstellen. Die Syntax ist folgende:
% \newcommand{neuer Befahl}{ein langer Befehl}
% Das folgende Beispiel fügt ein Bild mit bestimmten vorgegebenen Optionen ein:
\newcommand{\cImage}[1]{
	\begin{figure}[h!]
		\centering
		\includegraphics[width=0.50\textwidth]{#1}
	\end{figure}
}
% #1 ist dabei ein Parameter, den man \cImage übergeben muss. In 10_Titelseite.tex wird dieser Befehl verwendet. Der Parameter ist dort Bilder/titelseite.jpg.
% Benötigt man keine Parameter, dann lässt man [1] weg. Werden zusätzliche Parameter benötigt, dann kann man die Zahl auf maximal 9 erhöhen.

% Ausrichtung der Tabellenspalten
\newcolumntype{,}[1]{D{,}{,}{#1}}       % , in Tabellen untereinander stellen
\newcolumntype{p}{D{p}{\pm}{-1}}        % +- in Tabellen untereinander stellen

% Redefinitionen
\let\originalleft\left
\let\originalright\right
\renewcommand{\left}{\mathopen{}\mathclose\bgroup\originalleft}
\renewcommand{\right}{\aftergroup\egroup\originalright}

\let\rightarrowold\rightarrow
\let\leftarrowold\leftarrow
\renewcommand{\rightarrow}{\ensuremath{\rightarrowold}}
\renewcommand{\leftarrow}{\ensuremath{\leftarrowold}}

\let\equivold\equiv
\newcommand{\ident}{\equivold}
\renewcommand{\equiv}{\Leftrightarrow}
\renewcommand{\implies}{\Rightarrow}
\renewcommand{\impliedby}{\Leftarrow}

\renewcommand{\Re}{\operatorname{Re}}
\renewcommand{\Im}{\operatorname{Im}}

\newcommand{\thetaold}{\theta}
\newcommand{\phiold}{\phi}
\newcommand{\epsilonold}{\epsilon}
\renewcommand{\theta}{\vartheta}
\renewcommand{\phi}{\varphi}
\renewcommand{\epsilon}{\varepsilon}


% % %
\newcommand{\captionf}[1]{\caption[#1]{#1 \footnotemark}}
\newcommand{\NN}{\mathbb{N}}
\newcommand{\ZZ}{\mathbb{Z}}
\newcommand{\QQ}{\mathbb{Q}}
\newcommand{\RR}{\mathbb{R}}
\newcommand{\CC}{\mathbb{C}}
\newcommand{\bigO}{\mathcal{O}}
\newcommand{\abs}[1]{\left\lvert #1 \right\rvert}
\newcommand{\norm}[1]{\left\lVert#1\right\rVert}
\newcommand{\seq}[1]{\left (#1\right)}
\newcommand{\series}[3]{\sum_{#1 = #2}^\infty #3}
\newcommand{\arsinh}{\operatorname{arsinh}}
\newcommand{\arcosh}{\operatorname{arcosh}}
\newcommand{\artanh}{\operatorname{artanh}}
\newcommand{\const}{\text{const.}}
\newcommand{\pvec}[1]{\vec{#1}\mkern2mu\vphantom{#1}}

\makeatletter
\newcommand*{\diff}%
		{\@ifnextchar^{\DIfF}{\DIfF^{}}}
\def\DIfF^#1{%
		\mathop{\mathrm{\mathstrut d}}%
			\nolimits^{#1}\gobblespace
}
\def\gobblespace{%
        \futurelet\diffarg\opspace}
\def\opspace{%
        \let\DiffSpace\!%
        \ifx\diffarg(%
                \let\DiffSpace\relax
        \else
                \ifx\diffarg[%
                        \let\DiffSpace\relax
                \else
                        \ifx\diffarg\{%
                                \let\DiffSpace\relax
                        \fi\fi\fi\DiffSpace}

\newcommand*{\pderiv}[3][]{\frac{\partial^{#1}#2}{\partial #3^{#1}}}
\newcommand*{\deriv}[3][]{\frac{\diff^{#1}#2}{\diff #3^{#1}}}
%\NewDocumentCommand\@fehlerfortpflanzung:deriv{m m}{%
%	\left( \pderiv{#1}{#1} \right)^2
%}
%\NewDocumentCommand\@fehlerfortpflanzung:final{m}{%
%	#1
%}
%\NewDocumentCommand\fehlerfortpflanzung{m >{\SplitList{;}} m}{%
%	\begin{aligned}
%		\displaystyle\Delta #1 &= \sqrt{\ProcessList{#2}{\@fehlerfortpflanzung:deriv{#1}}}\\
%		&= \sqrt{\ProcessList{#2}{\@fehlerfortpflanzung:final}}
%	\end{aligned}
%}
\NewDocumentCommand\set{mg}{%
	\ensuremath{\left\{ #1 \IfNoValueTF{#2}{}{\;\middle|\; #2} \right\}}
}
\NewDocumentCommand\ceil{s O{} m}{%
	\IfBooleanTF{#1} % starred
		{#2\lceil#3#2\rceil} % \ceil*[..]{..}
		{\left\lceil#3\right\rceil} % \ceil[..]{..}
}
\NewDocumentCommand\floor{s O{} m}{%
	\IfBooleanTF{#1} % starred
		{#2\lfloor#3#2\rfloor} % \floor*[..]{..}
		{\left\lfloor#3\right\rfloor} % \floor[..]{..}
}
\NewDocumentCommand\colvec{mmg}{
	\ensuremath{
	\begin{pmatrix}
		#1 \\
		#2 \\
		\IfValueTF{#3}{#3 \\}{}
	\end{pmatrix}}
}
