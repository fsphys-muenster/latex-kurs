% Der Befehl \newcommand kann auch benutzt werden um Variablen zu definieren:

% Nummer laut Praktikumsheft:
    \newcommand{\varNum}{W1}
% Name laut Praktikumsheft:
    \newcommand{\varName}{Stirling-Motor}
% Datum der Durchführung:
    \newcommand{\varDate}{2014-05-13}
% Autoren des Protokolls:
    \newcommand{\varAutor}{Alexandra Everwand, Simon May}
% Nummer der eigenen Gruppe (z.B. "1mo"):
    \newcommand{\varGruppe}{Gruppe 14}
% E-Mail-Adressen der Autoren:
    \newcommand{\varEmail}{a.everwand@uni-muenster.de\\simon.may@uni-muenster.de}
% E-Mail-Adresse anzeigen (true/false):
    \newcommand{\varZeigeEmail}{true}
% Kopfzeile anzeigen (true/false):
    \newcommand{\varZeigeKopfzeile}{true}
% Inhaltsverzeichnis anzeigen (true/false):
    \newcommand{\varZeigeInhaltsverzeichnis}{true}
% Literaturverzeichnis anzeigen (true/false):
    \newcommand{\varZeigeLiteraturverzeichnis}{true}


